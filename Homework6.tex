\documentclass{article}
\usepackage{graphicx}
\usepackage{float}
\usepackage{subfigure} 
\usepackage{amsmath}
\usepackage{amssymb}
\usepackage{listings}
\usepackage{color}
\usepackage{seqsplit}
\definecolor{dkgreen}{rgb}{0,0.6,0}
\definecolor{gray}{rgb}{0.5,0.5,0.5}
\definecolor{mauve}{rgb}{0.58,0,0.82}

\lstset{frame=tb,
  language=C++,
  aboveskip=3mm,
  belowskip=3mm,
  showstringspaces=false,
  columns=flexible,
  basicstyle={\small\ttfamily},
  numbers=none,
  numberstyle=\tiny\color{gray},
  keywordstyle=\color{blue},
  commentstyle=\color{dkgreen},
  stringstyle=\color{mauve},
  breaklines=true,
  breakatwhitespace=true,
  tabsize=3
}
\author{Wenye Xiong 2023533141}
\title{Homework 5}
\begin{document}
\maketitle
\section{Problem 1}
\\ \hspace*{\fill} \\
First, we will find the number of T-Routes from $A(0,0)$ to $B(7,5)$\\
\\ \hspace*{\fill} \\
This number is $\frac{(b-a)!}{(\frac{b-a}{2} + \frac{\beta - \alpha}{2})! (\frac{b-a}{2} - \frac{\beta - \alpha}{2})!}$, where a=0, b=7, $\alpha$=0, $\beta$=5. And we can get 7 is the number of T-Routes.\\
\\ \hspace*{\fill} \\
Then, we will list all the T-Routes.\\
\\ \hspace*{\fill} \\
$\overline{(0,0),(1,-1),(2,0),(3,1),(4,2),(5,3),(6,4),(7,5)}$ \ $\overline{(0,0),(1,1),(2,0),(3,1),(4,2),(5,3),(6,4),(7,5)}$ \ $\overline{(0,0),(1,1),(2,2),(3,1),(4,2),(5,3),(6,4),(7,5)}$ \ $\overline{(0,0),(1,1),(2,2),(3,3),(4,2),(5,3),(6,4),(7,5)}$ \ $\overline{(0,0),(1,1),(2,2),(3,3),(4,4),(5,3),(6,4),(7,5)}$ \ $\overline{(0,0),(1,1),(2,2),(3,3),(4,4),(5,5),(6,4),(7,5)}$ \ $\overline{(0,0),(1,1),(2,2),(3,3),(4,4),(5,5),(6,6),(7,5)}$ \ \\
\\ \hspace*{\fill} \\
\newpage
\section{Problem 2}
\\ \hspace*{\fill} \\
To show that if A, B satisfy the T-condition, then there is a T-route from A to B, we can simply offer a possible T-Route.\\
\\ \hspace*{\fill} \\
Consider b-a steps, for the first $|\beta-\alpha|$ steps, we move in the direction of $\beta$-$\alpha$. That is if $\beta$-$\alpha$ is positive, we move in the direction of upper right, otherwise we move in the direction of lower right.\\
\\ \hspace*{\fill} \\
After the first $|\beta-\alpha|$ steps, we are now at (a+$|\beta-\alpha|$, $\beta$). \\
\\ \hspace*{\fill} \\
Because A, B satisfy the T-condition, a+$|\beta-\alpha|$ is smaller than b. For the next b-(a+$|\beta-\alpha|$) steps, we take two steps as a unit: For each unit, we move in the direction of upper right for the first step, and move in the direction of lower right for the second step.\\
\\ \hspace*{\fill} \\
Because $2|(b - a + \beta - \alpha)$, we also have $2|(b - a - \beta + \alpha)$. So $2|b-(a+|\beta-\alpha|)$, and we can take the next b-(a+$|\beta-\alpha|$) steps as $\frac{b-(a+|\beta-\alpha|)}{2}$ units.\\
\\ \hspace*{\fill} \\
For each units, we are actually moving in the direction of right for two steps. So after b-(a+$|\beta-\alpha|$) steps, we are now at (b, $\beta$), which is B.\\
\\ \hspace*{\fill} \\
This is a T-route. So we have shown that if A, B satisfy the T-condition, then there is a T-route from A to B.\\
\\ \hspace*{\fill} \\
\newpage
\section{Problem 3}
\\ \hspace*{\fill} \\
We are very clear that $x_1$ can only be 0. So we can take every $x_i $ as $x_{i-1}$ and throw away the stupid $x_1$.After that the system is reduced to:\\
\\ \hspace*{\fill} \\
\begin{align*}
  \begin{cases}
    $x_1 + x_2 + x_3 +....+ x_{2n} = n$\\
    $x_1 + x_2 + x_3 +....+ x_{i-1} < \frac{i}{2} $\\
    $x_i \in \{0,1\}$ \\
  \end{cases}
\end{align*} 
\\ \hspace*{\fill} \\
Further more, for the second case, $<\frac{i}{2}$ is just equal to $\leq \frac{i-1}{2}$, since the sum of $x_i$ can only be an integer.\\
\\ \hspace*{\fill} \\
So we can rewrite the system as:\\
\\ \hspace*{\fill} \\
\begin{align*}
  \begin{cases}
    $x_1 + x_2 + x_3 +....+ x_{2n} = n$\\
    $x_1 + x_2 + x_3 +....+ x_{i} \leq \frac{i}{2} $\\
    $x_i \in \{0,1\}$ \\
  \end{cases}
\end{align*}
\\ \hspace*{\fill} \\
That is just the typical case of Catalan number. So the number of solutions is $C_{2n}$, which is $\frac{(2n)!}{(n+1)!n!}$.\\ 
\\ \hspace*{\fill} \\
\newpage
\section{Problem 4}
\\ \hspace*{\fill} \\
Consider $y_i = x_i - 1$, then we have $y_1 + y_2 + y_3 +....+ y_n = r - n$, where $y_i \geq 0$ and is an integer.\\
\\ \hspace*{\fill} \\
This corresponds to a choice of where to place n-1 addition signs in a row of r-n ones.\\
\\ \hspace*{\fill} \\
For example, let n=3 and r=6, then we have 111, $(1,1,1)$ is 1+1+1, and $(0,2,1)$ is +11+1\\
\\ \hspace*{\fill} \\
So totally we have r-1 signs(1 and +), and we need to choose n-1 places of signs to be +. This is just a set with r-1 elements $A=\{ (n-1) \cdot +, (r-n) \cdot 1\}$, so the number of solutions is 
\begin{pmatrix}
  r-1\\
  n-1\\
\end{pmatrix}
\\ \hspace*{\fill} \\
\newpage
\section{Problem 5}
Let U = $\{u_1, u_2, u_3, ..., u_n\}, V = \{v_1, v_2, v_3, ..., v_n\} $be two sets of n elements each.\\
\\ \hspace*{\fill} \\
Consider the set X = $\{ (A,B,C): A \subseteq U, |A| = 1, B \subseteq U \cup V, |B| = n-1, C \subseteq U \cup V, |C| = n. |A \cap B| = 0, |A \cap C| = 0, |B \cap C| = 0.\}$\\
\\ \hspace*{\fill} \\
If we choose A, then choose B, the rest are C, then: $|X| = n \cdot 
\begin{pmatrix}
  2n-1\\
  n-1\\
\end{pmatrix}$\\
\\ \hspace*{\fill} \\
Or, we assume that $|A \cup B \cap U| = r$, where r can be any integer from 1 to n. Then of course $|C \cap V| = r$, that means there are r elements in U that are in A or B, and there are r elements in V that are in C. We first choose $A \cup B$ from U, then we choose A from $A \cup B$. Lastly, we choose C from V. Then we have $|X| = \sum_{1}^{n} r
\begin{pmatrix}
  n\\
  r\\
\end{pmatrix} \cdot
\begin{pmatrix}
  n\\
  r\\
\end{pmatrix} $\\
\\ \hspace*{\fill} \\
So in conclusion, $\sum_{1}^{n}r \cdot 
\begin{pmatrix}
  n\\
  r\\
\end{pmatrix} \cdot
\begin{pmatrix}
  n\\
  r\\
\end{pmatrix} 
 = n \cdot 
\begin{pmatrix}
  2n-1\\
  n-1\\
\end{pmatrix}$\\
\\ \hspace*{\fill} \\
\newpage
\section{Problem 6}
\\ \hspace*{\fill} \\
$a_n = \sum_{k=s}^{n} (-1)^{n-k}
\begin{pmatrix}
  n\\
  k\\
\end{pmatrix} b_k$\\
\\ \hspace*{\fill} \\
Then we have:$\sum_{k=s}^{n} 
\begin{pmatrix}
  n\\
  k\\
\end{pmatrix} a_k = \sum_{k=s}^{n} 
\begin{pmatrix}
  n\\
  k\\
\end{pmatrix}\sum_{i=s}^{k} (-1)^{k-i}
\begin{pmatrix}
  k\\
  i\\
\end{pmatrix}b_i$
$=\sum_{i=s}^{n} \sum_{k=i}^{n} (-1)^{k-i}
\begin{pmatrix}
  n\\
  k\\
\end{pmatrix}
\begin{pmatrix}
  k\\
  i\\
\end{pmatrix}b_i$\\
\\ \hspace*{\fill} \\
Because $\sum_{k=i}^{n} (-1)^{k-i}
\begin{pmatrix}
  n\\
  k\\
\end{pmatrix}
\begin{pmatrix}
  k\\
  i\\
\end{pmatrix} = \sum_{k=i}^{n} (-1)^{k-i}
\begin{pmatrix}
  n\\
  i\\
\end{pmatrix}
\begin{pmatrix}
  n-i\\
  k-i\\
\end{pmatrix} = 
\begin{pmatrix}
  n\\
  i\\
\end{pmatrix} \sum_{k=i}^{n} (-1)^{k-i} 
\begin{pmatrix}
  n-i\\
  k-i\\
\end{pmatrix} = 
\begin{pmatrix}
  n\\
  i\\
\end{pmatrix} \sum_{t=0}^{n-i} (-1)^t 
\begin{pmatrix}
  n-i\\
  t\\
\end{pmatrix} = 
\begin{cases}
  1,\ n=i\\
  0,\ n > i\\
\end{cases}$\\
\\ \hspace*{\fill} \\
So we have $\sum_{k=s}^{n} 
\begin{pmatrix}
  n\\
  k\\
\end{pmatrix} a_k =\sum_{i=s}^{n} \sum_{k=i}^{n} (-1)^{k-i}
\begin{pmatrix}
  n\\
  k\\
\end{pmatrix}
\begin{pmatrix}
  k\\
  i\\
\end{pmatrix}b_i = b_n$
\end{document}