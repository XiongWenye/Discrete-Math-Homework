\documentclass{article}
\usepackage{graphicx}
\usepackage{float}
\usepackage{subfigure} 
\usepackage{amsmath}
\usepackage{amssymb}
\usepackage{listings}
\usepackage{color}
\usepackage{seqsplit}
\definecolor{dkgreen}{rgb}{0,0.6,0}
\definecolor{gray}{rgb}{0.5,0.5,0.5}
\definecolor{mauve}{rgb}{0.58,0,0.82}

\lstset{frame=tb,
  language=C++,
  aboveskip=3mm,
  belowskip=3mm,
  showstringspaces=false,
  columns=flexible,
  basicstyle={\small\ttfamily},
  numbers=none,
  numberstyle=\tiny\color{gray},
  keywordstyle=\color{blue},
  commentstyle=\color{dkgreen},
  stringstyle=\color{mauve},
  breaklines=true,
  breakatwhitespace=true,
  tabsize=3
}
\author{Wenye Xiong 2023533141}
\title{Homework 5}
\begin{document}
\maketitle
\section{Problem 1}
\\ \hspace*{\fill} \\
We need to find all integers i such that $g^i$ is a generator of $\mathbb{Z}_p^*$, where $p = 107$ and $g = 2$.\\
\\ \hspace*{\fill} \\
First, we know that the order of $\mathbb{Z}_p^*$ is $p-1 = 106$.\\
\\ \hspace*{\fill} \\
Then, we know that the order of $g^i$ is $\frac{106}{gcd(i, 106)}$.\\
\\ \hspace*{\fill} \\
So, $g^i$ is a generator of $\mathbb{Z}_p^*$ iff $gcd(i, 106) = 1$.\\
That is all odd integer i such that $1 \leq i \leq 106$ except 53.\\
\\ \hspace*{\fill} \\
To verify my provement, I wrote a program.\\
\begin{lstlisting}
#include <bits/stdc++.h>
using namespace std;
int powmod(int a, int b, int m)
{
    int res = 1;
    while (b--)
    {
        res = (res * a) % m;
    }
    return res;
}
bool isgenerator(int g, int p)
{
    set<int> s;
    for (int i = 1; i < p; i++)
    {
        s.insert(powmod(g, i, p));
    }
    return s.size() == p - 1;
}
int main()
{
    for (int i = 1; i <= 106; i++)
    {
        if (isgenerator(powmod(2, i, 107), 107))
        {
            cout << i << endl;
        }
    }
    return 0;
}
\end{lstlisting}

According to the result of the program, we can see that all the integers i such that $g^i$ is a generator of $\mathbb{Z}_p^*$ are:\\
1 3 5 7 9 11 13 15 17 19 21 23 25 27 29 31 33 35 37 39 41 43 45 47 49 51 55 57 59 61 63 65 67 69 71 73 75 77 79 81 83 85 87 89 91 93 95 97 99 101 103 105\\
\\ \hspace*{\fill} \\
\newpage
\section{Problem 2}
We know that the order of $\mathbb{Z}_p^*$ is $p-1 = 2q$.\\
The order of a generator a of $\mathbb{Z}_p^*$ is also $2q$, which means that the smallest integer l to satisfy $a^l = 1$ is 2q.\\
\\ \hspace*{\fill} \\
For $g^i$, because $\frac{(p-1) \cdot i}{gcd(i, p-1)} \geq (p-1) $, so ${(g^i)}^{\frac{p-1}{gcd(i, p-1)}} = g^{i \cdot \frac{p-1}{gcd(i, p-1)}} = g^i^k$, where k is an integer no less than 1.\\
\\ \hspace*{\fill} \\
So, ${(g^i)}^{\frac{p-1}{gcd(i, p-1)}} = 1$, that means the order of $g^i$ is no greater than $\frac{p-1}{gcd(i, p-1)}$.\\
\\ \hspace*{\fill} \\
So, $g^i$ is a generator of $\mathbb{Z}_p^*$ only if $gcd(i, p-1) = 1$.\\
\\ \hspace*{\fill} \\
Then let's prove that if $gcd(i, 2q) = 1$, then $g^i$ is a generator of $\mathbb{Z}_p^*$.\\
\\ \hspace*{\fill} \\
Suppose that we can find an integer m such that $g^m$ is not a generator but $gcd(m, p-1) = 1$.\\
\\ \hspace*{\fill} \\
Take n as the least integer to satisfy ${(g^m)}^n = 1$.\\
\\ \hspace*{\fill} \\
Also, because ${(g^m)}^{(p-1)}} = 1$, we have $n | (p-1)$. Which means n is 2 or q.\\
\\ \hspace*{\fill} \\
We have $g^{m \cdot n} = 1$, which means $m \cdot n$ is a multiple of p-1.\\
\\ \hspace*{\fill} \\
Because $gcd(m, p-1) = 1$, and n can only be 2 or q, so mn can only be a multiple of 2 or a multiple q, but not a multiple of 2q, which is p-1. And this is just a contradiction.\\
\\ \hspace*{\fill} \\
So we have proved that $g^i$ is a generator of $\mathbb{Z}_p^*$ iff $gcd(i, 2q) = 1$.\\
\\ \hspace*{\fill} \\
Of all the integers i such that $1 \leq i \leq 2q$, the number of integers i that satisfy $gcd(i, 2q) = 1$ is $\phi(2q)$, which is $q-1$.\\
\\ \hspace*{\fill} \\
So the number of generators of $\mathbb{Z}_p^*$ is $q-1$.\\
\\ \hspace*{\fill} \\
\newpage
\section{Problem 3}
\begin{lstlisting}
p=1797693134862315907729305190789024733617976978942306572734300811577326758055009
631327084773224075360211201138798713933576587897688144166224928474306394741243777
678934248654852763022196012460941194530829520850057688381506823424628814739131105
40827237163350510684586298239947245938479716304835356329624227998859

A=1129835751630026189475896666667354281816845178451448750969029100664347239526230
166033932125012141273999088232234924787259712660427548927981777812675128216074705
452830594726890347313130276198642286884664382583275520454375902037906355067286037
74799021127049872571983254506993921153718739796769296097404717448108

B=1117727678052102394963651916915168810433949881962970620138536466745747434010427
364473288861564296291926916015263983660880127367494546266862814675792056750844619
894945132946240660741372479130373300404872753469132533457334297677819009771026871
85378411660147190296412313303321533586102552123457499563789255321369

a=0
b=0
for i in range(1, 10000):
    if pow(3, i, p) == A:
        a = i
        print(i)
        break

for i in range(1, 10000):
    if pow(3, i, p) == B:
        b = i
        print(i)
        break

print(pow(A, b, p))
print(pow(B, a, p))
\end{lstlisting}
The result: a = 9385, b = 3083\\
The output of Alice and Bob is \seqsplit{10828112783453462381041707802056149866596392072243903940987459672779260675319522663099080388770903982546250524992420350200207624327420612300170620802665302905750045777684348125827484365007590718638373187936889967309324722655294992225815410914105072210725045953105019352457540772995508978315699107247398350128}\\
\\ \hspace*{\fill} \\
\newpage
\section{Problem 4}
For x in $[1,2]$, $f(x) = 10^{1/(x-1)}$, that makes up $[10,+ \infty)$\\
\\ \hspace*{\fill} \\
For x in $(5,6]$, let $f(6) = 8 - \frac{1}{2}, f(6 - 2^{-n}) = 8 - 2^{-n-1}, n = 1, 2, 3, .....$\\
\\ \hspace*{\fill} \\
$f(x) = x + 2$, for all other $x \in (5, 6]$ And we have constructed a bijection between $(5,6]$ and $(7,8)$\\
\\ \hspace*{\fill} \\
Now our mission is to construct a bijection between $[3,4)$ and $(9,10)$.\\
\\ \hspace*{\fill} \\
Let $f(3) = 9 + \frac{1}{2}, f(3 + 2^{-n}) = 9 + 2^{-n-1}, n = 1, 2, 3, .....$\\
\\ \hspace*{\fill} \\
$f(x) = x + 6$, for all other $x \in [3, 4)$\\
\\ \hspace*{\fill} \\
Then we can see that $f(x)$ is a bijection between $[1,2] \cup [3,4) \cup (5,6]$ and $(7,8) \cup (9,\infty)$\\
\\ \hspace*{\fill} \\
\newpage
\section{Problem 5}
Suppose that $|{(a_1, a_2, a_3, . . .) : a_i \in {1, 2, 3}\ for\ all\ i = 1, 2, 3, . . .}| = |\mathbb{Z}^+|$\\
Denote ${(a_1, a_2, a_3, . . .) : a_i \in {1, 2, 3}\ for\ all\ i = 1, 2, 3, . . .}$ as S, then we have a bijection between f: \mathbb{Z}^+ \rightarrow S.\\
\\ \hspace*{\fill} \\
$f(1) = {a_{11}, a_{12}, a_{13}, . . .}$\\
$f(2) = {a_{21}, a_{22}, a_{23}, . . .}$\\
$f(3) = {a_{31}, a_{32}, a_{33}, . . .}$\\

\dots\\
$f(n) = {a_{n1}, a_{n2}, a_{n3}, . . .}$\\
\\ \hspace*{\fill} \\
Then we let $a_i = 1$ if $a_{ii} \neq 1$, $a_i = 2$ if $a_{ii} = 1$.\\
\\ \hspace*{\fill} \\
Obviously, set s = {a1, a2, a3, . . .} is in S, but has no preimage in \mathbb{Z}^+, since $s \neq f(i)$ for every i = 1,2,3,...n. That means f can't be a bijection\\
\\ \hspace*{\fill} \\
So $|{(a_1, a_2, a_3, . . .) : a_i \in {1, 2, 3}\ for\ all\ i = 1, 2, 3, . . .}| \neq |\mathbb{Z}^+|$.\\
\newpage
\section{Problem 6}
Suppose that $|A| = k$, because $A \cap B = \emptyset$, B can only be the subsets of X taken these k elements away.\\
\\ \hspace*{\fill} \\
For A, the number of ways to choose k elements X is $C_n^k$.\\
\\ \hspace*{\fill} \\
For B, we want to know the number of subsets of X taken k elements away, that is a normal set with n-k elements. And the number is $2^{n-k}$.\\
\\ \hspace*{\fill} \\
And the total number of sets {A,B} is just to sum up all the possibilities of k from 0 to n. But remember we also make many repeatition, since the set$\{A,B\}$ is just the same as $\{B,A\}$. So we need to divide this result by 2. But what is tricky here is the set $\{\emptyset,\emptyset \} $: we have only counted it once! So if we want the real answer, we have to add 1 before dividing 2.\\
\\ \hspace*{\fill} \\
So the total number of sets is $\frac{(\sum_{i=0}^{n}C_n^i2^{n-i})+1}{2} $, which can be further simplified as $\frac{(3^n)+1}{2} $\\
\\ \hspace*{\fill} \\
So the final result is $\frac{(3^n)+1}{2} $
\end{document}