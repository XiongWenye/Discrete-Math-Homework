\documentclass{article}
\usepackage{graphicx}
\usepackage{float}
\usepackage{subfigure} 
\usepackage{amsmath}
\usepackage{amssymb}
\usepackage{listings}
\usepackage{color}
\usepackage{seqsplit}
\definecolor{dkgreen}{rgb}{0,0.6,0}
\definecolor{gray}{rgb}{0.5,0.5,0.5}
\definecolor{mauve}{rgb}{0.58,0,0.82}

\lstset{frame=tb,
  language=C++,
  aboveskip=3mm,
  belowskip=3mm,
  showstringspaces=false,
  columns=flexible,
  basicstyle={\small\ttfamily},
  numbers=none,
  numberstyle=\tiny\color{gray},
  keywordstyle=\color{blue},
  commentstyle=\color{dkgreen},
  stringstyle=\color{mauve},
  breaklines=true,
  breakatwhitespace=true,
  tabsize=3
}
\author{Wenye Xiong 2023533141}
\title{Homework 8}
\begin{document}
\maketitle
\section{Problem 1}
The truth table for $ p \rightarrow \neg q \vee r \rightarrow (\neg r \rightarrow s \wedge p)$ is shown below:\\
\begin{table}[H]
\centering
\begin{tabular}{|c|c|c|c|c|}
\hline
$q$ & $r$ & $s$ & $p$ & $ p \rightarrow \neg q \vee r \rightarrow (\neg r \rightarrow s \wedge p)$\\
\hline
F & F & F & F & T\\
F & F & F & T & T\\
F & F & T & F & T\\
F & F & T & T & T\\
F & T & F & F & T\\
F & T & F & T & T\\
F & T & T & F & T\\
F & T & T & T & T\\
T & F & F & F & F\\
T & F & F & T & T\\
T & F & T & F & T\\
T & F & T & T & T\\
T & T & F & F & T\\
T & T & F & T & T\\
T & T & T & F & T\\
T & T & T & T & T\\
\hline
\end{tabular}
\end{table}
\section{Problem 2}
(1): The truth table for the formula $((p \vee q) \wedge (r \vee s)) \rightarrow (((p \rightarrow q) \vee (p \rightarrow r)) \wedge ((q \rightarrow p) \vee (q \rightarrow p)))$ is:\\
\begin{table}[H]
\centering
\begin{tabular}{|c|c|c|c|c|c|}
\hline
$p$ & $q$ & $r$ & $s$ &  $((p \vee q) \wedge (r \vee s)) \rightarrow (((p \rightarrow q) \vee (p \rightarrow r)) \wedge ((q \rightarrow p) \vee (q \rightarrow p)))$\\ 
\hline
F & F & F & F & T\\
F & F & F & T & T\\
F & F & T & F & T\\
F & F & T & T & T\\
F & T & F & F & T\\
F & T & F & T & F\\
F & T & T & F & F\\
F & T & T & T & F\\
T & F & F & F & T\\
T & F & F & T & F\\
T & F & T & F & T\\
T & F & T & T & T\\
T & T & F & F & T\\
T & T & F & T & T\\
T & T & T & F & T\\
T & T & T & T & T\\
\hline
\end{tabular}
\end{table}
So the formula is a contingency.\\
(2): The truth table for the formula $(\neg (p \leftrightarrow q) \rightarrow ((p \wedge \neg q) \vee (\neg p \wedge q))) \vee r$ is:\\
\begin{table}[H]
\centering
\begin{tabular}{|c|c|c|c|c|}
\hline
$p$ & $q$ & $r$ & $\neg (p \leftrightarrow q) \rightarrow ((p \wedge \neg q) \vee (\neg p \wedge q)) \vee r$\\
\hline
F & F & F & T\\
F & F & T & T\\
F & T & F & T\\
F & T & T & T\\
T & F & F & T\\
T & F & T & T\\
T & T & F & T\\
T & T & T & T\\
\hline
\end{tabular}
\end{table}
So the formula is a tautology.\\
(3): The truth table for the formula $ ((p \rightarrow r) \wedge (q \rightarrow s) \wedge (p \vee q)) \rightarrow (r \vee s)$ is:\\
\begin{table}[H]
\centering
\begin{tabular}{|c|c|c|c|c|c|}
\hline
$p$ & $q$ & $r$ & $s$ & $((p \rightarrow r) \wedge (q \rightarrow s) \wedge (p \vee q)) \rightarrow (r \vee s)$\\
\hline
F & F & F & F & T\\
F & F & F & T & T\\
F & F & T & F & T\\
F & F & T & T & T\\
F & T & F & F & T\\
F & T & F & T & T\\
F & T & T & F & T\\
F & T & T & T & T\\
T & F & F & F & T\\
T & F & F & T & T\\
T & F & T & F & T\\
T & F & T & T & T\\
T & T & F & F & T\\
T & T & F & T & T\\
T & T & T & F & T\\
T & T & T & T & T\\
\hline
\end{tabular}
\end{table}
So the formula is a tautology.\\
\section{Problem 3}
(1): $(B \vee A) \rightarrow C$\\
(2): $((A \rightarrow C) \wedge (\neg A)) \rightarrow ((\neg B) \vee (\neg C))$\\
\section{Problem 4}
(1): A: "x**y is valid Python", B: "x is a numeric number", C: "y is a numeric number". The logical notation is: $A \leftrightarrow (B \wedge C)$.\\
\\ \hspace*{\fill} \\
(2): A: "x + y is valid Python", B: "x is a numeric number", C: "y is a numeric number", D: "x is a list", E: "y is a list". The logical notation is: $A \leftrightarrow ((B \wedge C) \vee (D \wedge E))$.\\
\\ \hspace*{\fill} \\
(3): A: "x * y is valid Python", B: "x is a numeric number", C: "y is a numeric number", D: "x is a list", E: "y is a list". The logical notation is: $A \leftrightarrow ((B \wedge C) \vee (D \wedge C) \vee (E \wedge B))$.\\
\\ \hspace*{\fill} \\
(4): A: "x * y is a list", B: "x * y is valid Python", C: "x is a numeric number", D: "y is a numeric number". The logical notation is: $(B \wedge (\neg (C \wedge D))) \rightarrow A$.\\
\\ \hspace*{\fill} \\
(5): A: "x + y is valid Python", B: "x ** y is valid Python", C: "x is a list". The logical notation is: $(A \wedge B) \rightarrow (\neg C)$.\\
\section{Problem 5}
(1): $(p \rightarrow \neg p) \rightarrow (p \rightarrow q)$\\
\\ \hspace*{\fill} \\
(2): $(p \rightarrow (\neg p \rightarrow p)) \rightarrow q$\\
\section{Problem 6}
(1): To prove that $A \oplus B \leftrightarrow \neg (A \leftrightarrow B)$ is a tautology, let's consider the truth table for $A \oplus B$ and $\neg (A \leftrightarrow B)$:\\
\begin{table}[H]
\centering
\begin{tabular}{|c|c|c|c|}
\hline
$A$ & $B$ & $A \oplus B$ & $\neg (A \leftrightarrow B)$\\
\hline
F & F & F & F\\
F & T & T & T\\
T & F & T & T\\
T & T & F & F\\
\hline
\end{tabular}
\end{table}
So $A \oplus B \leftrightarrow \neg (A \leftrightarrow B)$ is a tautology.\\
\\ \hspace*{\fill} \\
(2): A = "You may contact me by phone", B = "You may contact me by email", C = "I am available for an on-site interview on October 8th in Minneapolis", D = "I am available for an on-site interview on October 8th in Hong Kong". The logical notation is: $(A \vee B) \wedge (C \oplus D)$.\\
\section{Problem 7}
$A_1 = p \wedge q \wedge r$\\
$A_2 = p \wedge q \wedge \neg r$\\
$A_3 = p \wedge \neg q \wedge r$\\
$A_4 = p \wedge \neg q \wedge \neg r$\\
$A_5 = \neg p \wedge q \wedge r$\\
$A_6 = \neg p \wedge q \wedge \neg r$\\
$A_7 = \neg p \wedge \neg q \wedge r$\\
$A_8 = \neg p \wedge \neg q \wedge \neg r$\\
$A = A_2 \vee A_4 \vee A_6 \vee A_7 \vee A_8$\\
\section{Problem 8}
$A = (p \wedge q) \vee (\neg p \wedge q \wedge r)$, $B = (p \vee (q \wedge r)) \wedge (q \vee (\neg p \wedge r))$\\
\\ \hspace*{\fill} \\
$A \equiv (p \wedge q) \vee (\neg p \wedge q \wedge r) \\
\equiv (p \vee (\neg p \wedge q \wedge r)) \wedge (q \vee (\neg p \wedge q \wedge r))\\
 \equiv  (p \vee \neg p) \wedge (p \vee (q \wedge r)) \wedge (q \vee (\neg p \wedge r))\\
  \equiv (p \vee (q \wedge r)) \wedge (q \vee (\neg p \wedge r)) \equiv B$\\
\section{Problem 9}
(1): To show that $(p \rightarrow q) \wedge (r \rightarrow s) \rightarrow ((p \wedge r) \rightarrow (q \wedge s))$ is a tautology, we are going to rewrite this to T:\\
$(p \rightarrow q) \wedge (r \rightarrow s) \rightarrow ((p \wedge r) \rightarrow (q \wedge s))
\leftrightarrow (\neg p \vee q) \wedge (\neg r \vee s) \rightarrow \neg (p \wedge r) \vee (q \wedge s)\\
\leftrightarrow (\neg p \vee q) \wedge (\neg r \vee s) \rightarrow (\neg p \vee \neg r) \vee (q \wedge s)\\
\leftrightarrow (\neg p \vee q) \wedge (\neg r \vee s) \rightarrow (\neg p \vee \neg r \vee q) \wedge (\neg p \vee \neg r \vee s)\\
\leftrightarrow (\neg p \vee q) \wedge (\neg r \vee s) \rightarrow ((\neg p \vee q) \wedge (\neg r \vee s)) \vee (\neg p \vee \neg r)\\
\leftrightarrow \neg ((\neg p \vee q) \wedge (\neg r \vee s)) \vee ((\neg p \vee q) \wedge (\neg r \vee s)) \vee (\neg p \vee \neg r)\\
\leftrightarrow T \vee (\neg p \vee \neg r) \equiv T\\


$
(2): To show that $(p \rightarrow q) \wedge (r \rightarrow s) \rightarrow ((p \wedge r) \rightarrow (q \wedge s))$ is a tautology, we are going to rewrite this to T:\\
$
((p \vee q) \wedge (p \rightarrow r) \wedge (q \rightarrow r)) \rightarrow r\\
 \leftrightarrow ((p \vee q) \wedge ( \neg p \vee r) \wedge ( \neg q \vee r)) \rightarrow r\\
 \leftrightarrow  \neg((p \vee q) \wedge ( \neg p \vee r) \wedge ( \neg q \vee r)) \vee r\\
 \leftrightarrow  \neg(p \vee q) \vee  \neg( \neg p \vee r) \vee  \neg( \neg q \vee r) \vee r\\
 \leftrightarrow ( \neg p \wedge  \neg q) \vee ( \neg \neg p \wedge  \neg r) \vee ( \neg \neg q \wedge  \neg r) \vee r\\
 \leftrightarrow ( \neg p \wedge  \neg q) \vee (p \wedge  \neg r) \vee (q \wedge  \neg r) \vee r\\
 \leftrightarrow ( \neg p \wedge  \neg q) \vee (p \wedge  \neg r) \vee ((q \vee r) \wedge ( \neg r \vee r))\\
 \leftrightarrow ( \neg p \wedge  \neg q) \vee (p \wedge  \neg r) \vee ((q \vee r) \wedge T)\\
 \leftrightarrow ( \neg p \wedge  \neg q) \vee (p \wedge  \neg r) \vee q \vee r\\
 \leftrightarrow ( \neg p \wedge  \neg q) \vee (p \wedge  \neg r) \vee r \vee q\\
 \leftrightarrow ( \neg p \wedge  \neg q) \vee ((p \vee r) \wedge ( \neg r \vee r)) \vee q\\
 \leftrightarrow ( \neg p \wedge  \neg q) \vee ((p \vee r) \wedge T) \vee q\\
 \leftrightarrow ( \neg p \wedge  \neg q) \vee p \vee r \vee q\\
 \leftrightarrow (( \neg p \vee p) \wedge ( \neg q \vee p)) \vee r \vee q\\
 \leftrightarrow (T \wedge ( \neg q \vee p)) \vee r \vee q\\
 \leftrightarrow  \neg q \vee p \vee r \vee q\\
 \leftrightarrow  \neg q \vee q \vee p \vee r\\
 \leftrightarrow T \vee p \vee r\\
 \leftrightarrow T\\
$
\section{Problem 10}
You should ask: "What path would the other type of person tell me is the correct one?" And then always choose the opposite one to go.\\
P: The man is a knight, R: The way he told me is to turn left. Q: The correct way is to turn left. And here's the truth table for the problem:\\
\begin{table}[H]
\centering
\begin{tabular}{|c|c|c|}
\hline
P & Q & R\\
\hline
T & T & F\\
T & F & T\\
F & T & F\\
F & F & T\\
\hline
\end{tabular}
\end{table}

\end{document}