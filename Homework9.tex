\documentclass{article}
\usepackage{graphicx}
\usepackage{float}
\usepackage{subfigure} 
\usepackage{amsmath}
\usepackage{amssymb}
\usepackage{listings}
\usepackage{color}
\usepackage{seqsplit}
\definecolor{dkgreen}{rgb}{0,0.6,0}
\definecolor{gray}{rgb}{0.5,0.5,0.5}
\definecolor{mauve}{rgb}{0.58,0,0.82}

\lstset{frame=tb,
  language=C++,
  aboveskip=3mm,
  belowskip=3mm,
  showstringspaces=false,
  columns=flexible,
  basicstyle={\small\ttfamily},
  numbers=none,
  numberstyle=\tiny\color{gray},
  keywordstyle=\color{blue},
  commentstyle=\color{dkgreen},
  stringstyle=\color{mauve},
  breaklines=true,
  breakatwhitespace=true,
  tabsize=3
}
\author{Wenye Xiong 2023533141}
\title{Homework 9}
\begin{document}
\maketitle
\section{Problem 1}
Let $A = (\neg p \vee q) \wedge (r \rightarrow \neg q)$, $B = p \rightarrow \neg r$. We need to prove that $A \rightarrow B \equiv T$\\
\\ \hspace*{\fill} \\
$A \rightarrow B = (\neg p \vee q) \wedge (r \rightarrow \neg q) \rightarrow (p \rightarrow \neg r)$\\
$= \neg ((\neg p \vee q) \wedge (r \rightarrow \neg q)) \vee (p \rightarrow \neg r)$\\
$= (\neg (\neg p \vee q) \vee \neg (r \rightarrow \neg q)) \vee (p \rightarrow \neg r)$\\
$= ((p \wedge \neg q) \vee (r \wedge q)) \vee (p \rightarrow \neg r)$\\
$= (p \wedge \neg q) \vee (r \wedge q) \vee (\neg p \vee \neg r)$\\
$= (p \wedge \neg q) \vee (r \wedge q) \vee \neg p \vee \neg r$\\
$= (p \wedge \neg q) \vee \neg p \vee (r \wedge q) \vee \neg r$\\
$= \neg q \vee \neg p \vee q \vee \neg r$\\
$= T$\\
\section{Problem 2}
Let $A = (p \rightarrow q \vee r) \wedge (q \rightarrow s) \wedge (r \rightarrow \neg p)$, $B = p \rightarrow s$. We need to prove that $A \wedge \neg B \equiv F$\\
\\ \hspace*{\fill} \\
$A \wedge \neg B = (p \rightarrow q \vee r) \wedge (q \rightarrow s) \wedge (r \rightarrow \neg p) \wedge \neg (p \rightarrow s)$\\
$= (\neg p \vee (q \vee r)) \wedge (\neg p \vee s) \wedge (\neg r \vee \neg p) \wedge \neg (\neg p \vee s)$\\
$= (\neg p \vee q \vee r) \wedge (\neg p \vee s) \wedge (\neg r \vee \neg p) \wedge (p \wedge \neg s)$\\
$= (\neg p \vee q \vee r) \wedge (\neg p \vee s) \wedge (\neg r \vee \neg p) \wedge p \wedge \neg s$\\
$= (\neg p \vee q \vee r) \wedge (p \wedge s) \wedge (\neg r \vee \neg p) \wedge \neg s$\\
$= (\neg p \vee q \vee r) \wedge (\neg r \vee \neg p) \wedge p \wedge (s \wedge \neg s)$\\
$= (\neg p \vee q \vee r) \wedge (\neg r \vee \neg p) \wedge F$\\
$= F$\\
\section{Problem 3}
Let p be proposition "It rains", let q be proposition "it is foggy", let r be proposition "The sailing race will be held", and let t be proposition "The life saving demonstration will go on", and let u be proposition "The trophy will be awarded".\\
\\ \hspace*{\fill} \\
Building Arguments:\\
(1): $\neg u$ Premise\\
(2): $r \implies u$ Premise\\
(3): $\neg r$ Modus Tollens of (1) and (2)\\
(4): $(\neg p \vee \neg q) \implies (r \wedge t)$ Premise\\
(5): $\neg (r \wedge t) \implies \neg (\neg p \vee \neg q)$ Contrapositive of (4)\\
(6): $\neg r \vee \neg t \implies (p \wedge q)$ Equivalence of (5)\\
(7): $\neg r \vee \neg t$ Addition of (3)\\
(8): $p \wedge q$ Modus Ponens of (6) and (7)\\
(9): p Simplification of (8)\\


\section{Problem 4}
Premises: $p \wedge q \rightarrow r$, $\neg r \vee s$, $p \rightarrow \neg s$\\
\\ \hspace*{\fill} \\
Conclusion: $p \rightarrow \neg q$\\
\\ \hspace*{\fill} \\
(1): $p \wedge q \rightarrow r$ Premise\\
(2): $\neg r \vee s$ Premise\\
(3): $p \rightarrow \neg s$ Premise\\
(4): $\neg p \vee \neg s$ Equivalence of (3)\\
(5): $\neg p \vee \neg r$ Resolution of (2) and (4)\\
(6): $p \rightarrow \neg r$ Equivalence of (5)\\
(7): $r \rightarrow \neg p$ Contrapositive of (6)\\
(8): $(p \wedge q) \rightarrow \neg p$ Rule of Inference from (1) and (7)\\
(9): $\neg (p \wedge q) \vee \neg p$ Equivalence of (8)\\
(10): $\neg p \vee \neg q \vee \neg p$ De Morgan's Law of (9)\\
(11): $\neg p \vee \neg p \vee \neg q$ Commutative Law of (10)\\
(12): $\neg p \vee \neg q$ Idempotent Law of (11)\\
(13): $p \rightarrow \neg q$ Equivalence of (12)\\
\section{Problem 5}
Premises: $p_1 \rightarrow (q_1 \rightarrow r_1)$, $p_2 \rightarrow (q_2 \rightarrow r_2)$, $p_3 \rightarrow (q_3 \rightarrow r_3)$, $\cdots$, $p_n \rightarrow (q_n \rightarrow r_n)$, $q_1 \wedge q_2 \wedge q_3 \wedge \cdots \wedge q_n$\\
\\ \hspace*{\fill} \\
Conclusion: $(p_1 \rightarrow r_1) \wedge (p_2 \rightarrow r_2) \wedge (p_3 \rightarrow r_3) \wedge \cdots \wedge (p_n \rightarrow r_n)$\\
\\ \hspace*{\fill} \\
(1): $p_i \rightarrow (q_i \rightarrow r_i)$ Premise\\
(2): $q_1 \wedge q_2 \wedge q_3 \wedge \cdots \wedge q_n$ Premise\\
(3): $q_i$ Simplification of (2)\\
(4): $\neg p_i \vee r_i$ Equivalence of (1), (3)\\
(5): $p_i \rightarrow r_i$ Equivalence of (4)\\
(6): $(p_1 \rightarrow r_1) \wedge (p_2 \rightarrow r_2) \wedge (p_3 \rightarrow r_3) \wedge \cdots \wedge (p_n \rightarrow r_n)$ Conjunction of (5)\\
\section{Problem 6}
(a): True\\
(b): False\\
(c): True\\
(d): True\\
\section{Problem 7}
Let p be proposition "A is the suspect", q be proposition "A visited the victim's room", r be proposition "A leave before 2 am", s be proposition "The hotel staff saw A".\\
\\ \hspace*{\fill} \\
Building Arguments:\\
(1): $(q \wedge \neg r) \rightarrow p$ Premise\\
(2): $q$ Premise\\
(3): $r \rightarrow s$ Premise\\
(4): $\neg s$ Premise\\
(5): $\neg s \rightarrow \neg r$ Contrapositive of (3)\\
(6): $\neg r$ Modus Ponens of (4) and (5)\\
(7): $q \wedge \neg r$ Conjunction of (2) and (6)\\
(8): $p$ Modus Ponens of (1) and (7)\\
So we have proved A is the suspect.\\
\end{document}