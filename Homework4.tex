\documentclass{article}
\usepackage{graphicx}
\usepackage{float}
\usepackage{subfigure} 
\usepackage{amsmath}
\usepackage{amssymb}
\usepackage{listings}
\usepackage{color}
\definecolor{dkgreen}{rgb}{0,0.6,0}
\definecolor{gray}{rgb}{0.5,0.5,0.5}
\definecolor{mauve}{rgb}{0.58,0,0.82}

\lstset{frame=tb,
  language=C++,
  aboveskip=3mm,
  belowskip=3mm,
  showstringspaces=false,
  columns=flexible,
  basicstyle={\small\ttfamily},
  numbers=none,
  numberstyle=\tiny\color{gray},
  keywordstyle=\color{blue},
  commentstyle=\color{dkgreen},
  stringstyle=\color{mauve},
  breaklines=true,
  breakatwhitespace=true,
  tabsize=3
}
\author{Wenye Xiong 2023533141}
\title{Homework 1}
\begin{document}
\maketitle
\section{Problem 1}
\\ \hspace*{\fill} \\
$n_1 = 13, n_2 = 17, n_3 = 19, n_4 = 23$, these are all prime numbers.\\
\\ \hspace*{\fill} \\
So we let $n = n_1n_2n_3n_4 = 13\times17\times19\times23 = 96577$.\\
\\ \hspace*{\fill} \\
Then we let $N_1 = n/n_1 = 96577/13 = 7429$, $N_2 = n/n_2 = 96577/17 = 5681$, $N_3 = n/n_3 = 96577/19 = 5083$, $N_4 = n/n_4 = 96577/23 = 4199$.\\
\\ \hspace*{\fill} \\
Using the Extended Euclidean Algorithm, we can find $s_1, t_1, s_2, t_2, s_3, t_3, s_4, t_4$ such that $s_1n_1+t_1N_1 = 1$, $s_2n_2+t_2N_2 = 1$, $s_3n_3+t_3N_3 = 1$, $s_4n_4+t_4N_4 = 1$.\\
\\We get: $s_1 = 1143, t_1 = -2$; $s_2 = -2005, t_2 = 6$; $s_3 = -535, t_3 = 2$; $s_4 = 1278, t_4 = -7$.\\
\\ \hspace*{\fill} \\
So the one solution is:\\
$b = b_1N_1t_1 + b_2N_2t_2 + b_3N_3t_3 + b_4N_4t_4 = b_1\cdot7429\cdot(-2) + b_2\cdot5681\cdot6 + b_3\cdot5083\cdot2 + b_4\cdot4199\cdot(-7) = -14858b_1 + 34086b_2 + 10166b_3 - 29393b_4$.\\
\\ \hspace*{\fill} \\
Then $ x \in \mathbb{Z}$ is a solution iff $x \equiv b \pmod {96577}$. That is $[b]_{96577}$\\
\\ \hspace*{\fill} \\
\newpage
\section{Problem 2}
\\ \hspace*{\fill} \\
The first equation requires that $x = 6t + b_1.$\\
\\ \hspace*{\fill} \\
Take this into the second equation, we get $6t + b_1 \equiv b2 \pmod {15}$.\\
\\ \hspace*{\fill} \\
So $6t \equiv b_2 - b_1 \pmod {15}$.\\
\\ \hspace*{\fill} \\
Since $gcd(6, 15) = 3$, it is clear that $3|b_2 - b_1$.\\
\\ \hspace*{\fill} \\
Under this condition, let's solve the problem:\\
\\ \hspace*{\fill} \\
The first equation requires that $x = 6t_1 + b_1.$, while the second equation requires that $x = 15t_2 + b_2.$\\
\\ \hspace*{\fill} \\
So $6t_1 + b_1 = 15t_2 + b_2$.\\
\\ \hspace*{\fill} \\
And we get: $6t_1 - 15t_2 = b_2 - b_1$. Because $3|b_2 - b_1$, we can write it as: $2t_1 - 5t_2 = \frac{b_2 - b_1}{3}$.\\
\\ \hspace*{\fill} \\
Using the Extended Euclidean Algorithm, we can find $s, t$ such that $2s - 5t = 1$. The answer is $s = 3, t = 1$\\
\\ \hspace*{\fill} \\
So one solution is: \\
\centerline{$t_1 = 3\cdot\frac{b_2 - b_1}{3} = b_2 - b_1 , t_2 = \frac{b_2 - b_1}{3}, x = 6b_2 - 5b_1$}\\
\\ \hspace*{\fill} \\
Then $ x \in \mathbb{Z}$ is a solution iff $x \equiv 6b_2 - 5b_1 \pmod {30}$. That is $[6b_2 - 5b_1]_{30}$\\
\\ \hspace*{\fill} \\
\newpage
\section{Problem 3}
To prove $x \star y = xy − x − y + 2 $made (G,⋆) an Abelian Group, we need to prove the following four properties:\\
\\ \hspace*{\fill} \\
1. Closure: Both x and y are greater than 1, so we know that $ x \star y = xy - (x + y) + 1 + 1 = (x - 1) \cdot (y - 1) + 1$ is also greater than 1. So $x \star y $ is in G.\\ 
\\ \hspace*{\fill} \\
2. Associativity: Let x, y, z be three elements in G, $(x \star y) \star z = ((x - 1) \cdot (y - 1) + 1) \star z = ((x - 1) \cdot (y - 1) + 1 - 1) \cdot (z - 1) + 1 = (x - 1) \cdot ((y - 1) \cdot (z - 1) + 1 - 1) + 1  = x \star (y \star z)$.\\
\\ \hspace*{\fill} \\
3. Identity: Let e = 2, then $x \star e = x \star 2 = 2x - (x + 2) + 2 = x$, $ e \star x = 2 \star x = 2x - (x+2) + 2 = x.$\\
\\ \hspace*{\fill} \\
4. Inverse: Let $x \in G$, $ x^{-1} = 1 + \frac{1}{x-1}$, then $x \star x^{-1} = x \star (1 + \frac{1}{x-1}) = x \cdot (1 + \frac{1}{x-1}) - (x + 1 + \frac{1}{x-1}) + 2 = 2 = e $.\\
We have $x^{-1} = 1 + \frac{1}{x-1}$. And since $x > 1$, it is clear that $x^{-1}$ is also in G.\\
\\ \hspace*{\fill} \\
5. Commutativity: Let x, y be two elements in G, $x \star y = xy - (x + y) + 2 = yx - (y + x) + 2 = y \star x$.\\
\\ \hspace*{\fill} \\
So (G,⋆) is an Abelian Group.\\
\\ \hspace*{\fill} \\
\newpage
\section{Problem 4}
\\ \hspace*{\fill} \\
$\mathbb{Z}_{23}^{*}$ is a multiplicative Abelian group of order 22, so for any $a \in \mathbb{Z}_{23}^{*}$, $a^{22} = 1$.\\
\\ \hspace*{\fill} \\
Thus the order of any element in that group must divide 22, so all orders must be 1, 2, 11, or 22. \\
\\ \hspace*{\fill} \\
To find the generator, we are looking for an element whose order is 22.\\
\\ \hspace*{\fill} \\
To find them, we need only compute the second and 11th powers of each element modulo 23; when neither is 1, we have found an element of order 22.\\
\\ \hspace*{\fill} \\
These are not generators:\\
1 (of cause)\\
2 ($2^{11} \equiv 1 \pmod{23}$)\\
3 ($3^{11} \equiv 1 \pmod{23}$)\\
4 ($4^{11} \equiv 1 \pmod{23}$)\\
6 ($6^{11} \equiv 1 \pmod{23}$)\\
8 ($8^{11} \equiv 1 \pmod{23}$)\\
9 ($9^{11} \equiv 1 \pmod{23}$)\\
12 ($12^{11} \equiv 1 \pmod{23}$)\\
13 ($13^{11} \equiv 1 \pmod{23}$)\\
16 ($16^{11} \equiv 1 \pmod{23}$)\\
18 ($18^{11} \equiv 1 \pmod{23}$)\\
22 ($22^2 \equiv 1 \pmod{23}$)\\
\\ \hspace*{\fill} \\
In conclusion, these are the generators of $\mathbb{Z}_{23}^{*}$: 5, 7, 10, 11, 14, 15, 17, 19, 20, 21.\\
\\ \hspace*{\fill} \\
$f(X) = X^5 + 2X^4 + 5X^3 + 3X^2 + 12X + 18$\\
\\ \hspace*{\fill} \\
For this problem, I couldn't come up with a better solution, so I wrote a program in cpp to find the roots. Here's the code.\\
\begin{lstlisting}
#include <bits/stdc++.h>
using namespace std;
int poly(int x)
{
    return pow(x, 5) + 2 * pow(x, 4) + 5 * pow(x, 3) + 3 * pow(x, 2 ) + 12 * x + 18;
}
int main()
{
    for (int i = 1; i <= 23; i++)
    {
        if (poly(i) % 23 == 0)
        {
            cout << i << " is a root of the polynomial" << endl;
        }
    }
    return 0;
}
\end{lstlisting}
The roots of this polynomial are 3, 11, 16, 17, 20.\\
\\ \hspace*{\fill} \\
\newpage
\section{Problem 5}
\\ \hspace*{\fill} \\
According to the definition of order, o(a) is the least integer greater than 0 such that $a^{o(a)} = 1$. To prove that $o(a)|k$, let's suppose that there is a postive integer k such that $a^k = 1$ and $o(a) \nmid k$.\\
\\ \hspace*{\fill} \\
Then we can write k as $k = o(a)q + r$, where $0 < r < o(a)$.\\
\\ \hspace*{\fill} \\
So $a^k = a^{o(a)q + r} = (a^{o(a)})^q \cdot a^r = 1^q \cdot a^r = a^r$.\\
\\ \hspace*{\fill} \\
Since $a^k = 1$, we have $a^r = 1$.\\
\\ \hspace*{\fill} \\
But $0 < r < o(a)$, which contradicts the definition of order.\\
\\ \hspace*{\fill} \\
So $o(a)|k$.\\
\\ \hspace*{\fill} \\
For a multiplicative Abelian Group G, G = {1,2,3,...,m}, and for any $i \neq j$, $aa_i \neq aa_j$.\\
\\ \hspace*{\fill} \\
So $aa_1 \cdot aa_2 \cdot aa_3 \cdot ... \cdot aa_m = a_1a_2....a_m$.\\
Thus $a^m = 1$. Together with the previous result, the order of any group element must be a divisor of the group’s order.\\
\\ \hspace*{\fill} \\
\newpage
\section{Problem 6}
\\ \hspace*{\fill} \\
According to the definition of order, o(a) is the least integer greater than 0 such that $a^{o(a)} = 1$ and o(b) is the least integer greater than 0 such that $b^{o(b)} = 1$.\\
\\ \hspace*{\fill} \\
Let $m = o(a), n = o(b)$. Consider $(ab)^{mn}$, we have $(ab)^{mn} = a^{mn}b^{mn} = (a^m)^n(b^n)^m = 1^n1^m = 1$. That means $o(ab) | mn$\\
\\ \hspace*{\fill} \\
Consider an integer l such that $(ab)^l = 1$\\
\\ \hspace*{\fill} \\
Then we have $a^{l}b^{l} = 1$, which means $a^{l} = b^{-l}$ since $a^l b^l b^{-l} = 1 \cdot b^{-l}$\\
\\ \hspace*{\fill} \\
So $a^{nl} = b^{-nl} = (b^n)^{-l} = 1$, which means $o(a) | nl$ as we have proved in Problem 5.\\
\\ \hspace*{\fill} \\
Because $gcd(o(a),o(b))=1$, we have $o(a) | l$. And similarly, we get $o(b) | l$\\
\\ \hspace*{\fill} \\
Because $gcd(o(a),o(b))=1$, so $mn | l=o(ab)$ and with $o(ab) | mn$. In general, $o(ab) = o(a) \cdot o(b)$\\
\\ \hspace*{\fill} \\
So $o(ab) = o(a) \cdot o(b)$.\\
\end{document}