\documentclass{article}
\usepackage{graphicx}
\usepackage{float}
\usepackage{subfigure} 
\usepackage{amsmath}
\usepackage{amssymb}

\author{Wenye Xiong 2023533141}
\title{Homework 1}
\begin{document}
\maketitle

\section{Problem 1}
\\ \hspace*{\fill} \\
Assume the statement is false. This means there exists a real number $ x $ and a positive integer $ n $ such that:\\

\centerline{$\left\lfloor\dfrac{\left\lfloor{x}\right\rfloor}{n}\right\rfloor \neq \left\lfloor\dfrac{x}{n}\right\rfloor$}\\
\\ \hspace*{\fill} \\

$n>0$ and $ x-1 < \left\lfloor{x}\right\rfloor < x$, so we can write:\\

\centerline{$\left\lfloor\dfrac{x-1}{n}\right\rfloor < \left\lfloor\dfrac{\left\lfloor{x}\right\rfloor}{n}\right\rfloor < \left\lfloor\dfrac{x}{n}\right\rfloor$ } \\

\\ \hspace*{\fill} \\
Both of the left and right hand side of the equation are integers, and $n>1$, so we have the only choice:\\

\centerline{$\left\lfloor\dfrac{\left\lfloor{x}\right\rfloor}{n}\right\rfloor = \left\lfloor\dfrac{x}{n}\right\rfloor-1$}\\

Substituting this back into the original inequality, we get:\\
\\ \hspace*{\fill} \\
\centerline{$\left\lfloor\dfrac{x}{n}\right\rfloor \neq \left\lfloor\dfrac{x}{n}\right\rfloor-1$}
\\ \hspace*{\fill} \\
This is a contradiction, so the original statement is true.\\

\\ \hspace*{\fill} \\
\section{Problem 2}
\\ \hspace*{\fill} \\
Assume that gcd(a,b)=1 and gcd(a,c)=1. Then we have:\\
\\ \hspace*{\fill} \\
\centerline{$ax+by=1$}
\centerline{$au+cv=1$}
x,y,u,v are all integers.\\
\\ \hspace*{\fill} \\
Multiplying the first equation with cv, we get:\\
\\ \hspace*{\fill} \\
\centerline{$acvx+bcyv=cv$}
\\ \hspace*{\fill} \\
Add both sides of the equation with au, we get:\\
\\ \hspace*{\fill} \\
\centerline{$acvx+bcyv+au=cv+au$}
\\ \hspace*{\fill} \\
Which means:\\
\\ \hspace*{\fill} \\
\centerline{$a(cvx+u)+bc(yv)=1$}
\\ \hspace*{\fill} \\
So we have gcd(a,bc)=1 because as c,v,x,u,y,v are all integers, cvx+u,yv are also integers.\\
\\ \hspace*{\fill} \\
So we have proved that if gcd(a,b)=1 and gcd(a,c)=1, then gcd(a,bc)=1.\\
\\ \hspace*{\fill} \\
Next we will prove that if gcd(a,bc)=1, then gcd(a,b)=1 and gcd(a,c)=1.\\
\\ \hspace*{\fill} \\
Assume that gcd(a,bc)=1, then we have:\\
\\ \hspace*{\fill} \\
\centerline{$ax+bcy=1$}
\\ \hspace*{\fill} \\
We know that x,c,y are all integers, so we can write the equation as:\\
\\ \hspace*{\fill} \\
\centerline{$ax+b(cy)=1$}, where x,cy are integers.\\
\\ \hspace*{\fill} \\
So we have gcd(a,b)=1.\\
\\ \hspace*{\fill} \\
Similarly, we can prove that gcd(a,c)=1 by writing the equation as:\\
\\ \hspace*{\fill} \\
\centerline{$ax+c(by)=1$}, where x,by are integers.\\
\\ \hspace*{\fill} \\
So we have proved if gcd(a,bc)=1, then gcd(a,b)=1, gcd(a,c)=1\\
\\ \hspace*{\fill} \\
In conclusion, we have proved that gcd(a,b)=1 and gcd(a,c)=1 if and only if gcd(a,bc)=1.\\
\\ \hspace*{\fill} \\
\section{Problem 3}
\\ \hspace*{\fill} \\
\large$\alpha = \sum_{i=1}^{\infty} \left\lfloor\frac{n}{p^i}\right\rfloor$\\
\\ \hspace*{\fill} \\
\normalsize
Here's the proof:\\
The first term $ \left\lfloor\frac{n}{p}\right\rfloor$appears since we want to count the number of terms less than n and are multiples of p. We first assume that each of these contributes one p to n!\\
\\ \hspace*{\fill} \\
But then when we consider multiples of $p^2$, we are not multiplying just one p but two of these p to the product. So we now count the number of multiple of $p^2$ less than n and add them.\\
\\ \hspace*{\fill} \\
This is captured by the second term $\left\lfloor\frac{n}{p^2}\right\rfloor$. Repeat this to account for higher powers of p less than n, and we have $\alpha = \sum_{i=1}^{\infty} \left\lfloor\frac{n}{p^i}\right\rfloor$\\
I can write the sum from 1 to infinity because the terms after $\left\lfloor\frac{n}{p^{log_pn}}\right\rfloor$ are all 0\\
\\ \hspace*{\fill} \\
\section{Problem 4}
\\ \hspace*{\fill} \\
Assume that there exists a x such that we can find a n where $x^7=n$ and $x\notin \mathbb{Z}^+$.\\
\\ \hspace*{\fill} \\
And since x is a rational number, we can write x as $\dfrac{a}{b}$, where a and b are both positive integers and gcd(a,b)=1\\
\\ \hspace*{\fill} \\
So we have $\dfrac{a^7}{b^7}=n$\\
\\ \hspace*{\fill} \\
Which means $a^7$ and $b^7$ share a common factor $b^7$\\
\\ \hspace*{\fill} \\
This obviously contradicts with the fact that gcd(a,b)=1, according to the FTA. So if x is a rational number and exists $n \in \mathbb{Z}^+$, then $x \in \mathbb{Z}^+$\\
\\ \hspace*{\fill} \\
\section{Problem 5}
\\ \hspace*{\fill} \\
Because we have $a,b \in \mathbb{Z}, n \in \mathbb{Z}^+$ and $a \equiv b $ (mod n), we can easily find that n divides a-b\\
\\ \hspace*{\fill} \\
Let's define $\sum_{i=0}^{k} c_i a^i $ as $S_a$ and $\sum_{i=0}^{k} c_i b^i $ as $S_b$\\
\\ \hspace*{\fill} \\
Then we have $S_a-S_b = \sum_{i=0}^{k} c_i (a^i-b^i) $\\
\\ \hspace*{\fill} \\
We can factor out a-b from the right hand side of the equation, and we get:\\
\\ \hspace*{\fill} \\
$S_a-S_b = (a-b) \sum_{i=0}^{k} c_i (a^{i-1}+a^{i-2}b+...+b^{i-1})$\\
\\ \hspace*{\fill} \\
As we have proved that n divides a-b, we can write a-b as $n \times m$\\
\\ \hspace*{\fill} \\
So we have $S_a-S_b = n \times m \times \sum_{i=0}^{k} c_i (a^{i-1}+a^{i-2}b+...+b^{i-1})$\\
\\ \hspace*{\fill} \\
Which means $S_a-S_b$ is divisible by n. So we have proved that $S_a \equiv S_b$ (mod n)\\
\\ \hspace*{\fill} \\
\section{Problem 6}
\\ \hspace*{\fill} \\
Multiplying $u^2+uv+v^2$ with $u-v$, we get $u^3-v^3$\\
\\ \hspace*{\fill} \\
So we have $u^3-v^3 \equiv 0$ (mod 9)\\
\\ \hspace*{\fill} \\
Assume that u and v are not equal after mod 3:\\
\\ \hspace*{\fill} \\
If $u \in [0]_3$ and $v \notin [0]_3$, then $u^3-v^3 \equiv v^3 \equiv $ 2 or 1 (mod 3). The same when $v \in [0]_3$ and $u \notin [0]_3$\\
\\ \hspace*{\fill} \\
If $u \in [1]_3$ and $v \in [2]_3$, then $u^3-v^3 \equiv 2$ (mod 3)\\
Or if $u \in [2]_3$ and $v \in [1]_3$, then $u^3-v^3 \equiv 1$ (mod 3)\\
\\ \hspace*{\fill} \\
None of these cases can satisfy the original equation, so we have proved that $u \equiv v$ (mod 3).\\
\\ \hspace*{\fill} \\
Assume that $u,v \in [1]_3$, then we can write u,v as $3k+1,3l+1$\\
\\ \hspace*{\fill} \\
Substituting these into the original equation, we get:\\
\\ \hspace*{\fill} \\
$(3k+1)^2+(3k+1)(3l+1)+(3l+1)^2 \equiv 0$ (mod 9)\\
Unfolding the equation, we get:, 
\\ \hspace*{\fill} \\
\centerline{$9k^2+9k+9l^2+9l+9kl+3 \equiv 0$ (mod 9)}\\
\\ \hspace*{\fill} \\
Which is always false, so we have proved that $u,v \in [1]_3$ is not a solution.\\
\\ \hspace*{\fill} \\
Next, consider the case where $u,v \in [2]_3$, then we can write u,v as $3k+2,3l+2$\\
\\ \hspace*{\fill} \\
Substituting these into the original equation, we get:\\
\\ \hspace*{\fill} \\
$(3k+2)^2+(3k+2)(3l+2)+(3l+2)^2 \equiv 0$ (mod 9), 
Unfolding the equation, we get:\\
\\ \hspace*{\fill} \\
\centerline{$9k^2+18k+9l^2+18l+9kl+12 \equiv 0$ (mod 9)}\\
\\ \hspace*{\fill} \\
Which is always false, so we have proved that $u,v \in [2]_3$ is not a solution.\\
\\ \hspace*{\fill} \\
Then consider the last case where $u,v \in [0]_3$, then we can write u,v as $3k,3l$\\
\\ \hspace*{\fill} \\
Substituting these into the original equation, we get:\\
\\ \hspace*{\fill} \\
$(3k)^2+(3k)(3l)+(3l)^2 \equiv 0$ (mod 9), 
Unfolding the equation, we get:\\
\\ \hspace*{\fill} \\
\centerline{$9k^2+9kl+9l^2 \equiv 0$ (mod 9)}\\
\\ \hspace*{\fill} \\
Which is always true, so we have proved that $u,v \in [0]_3$ is a solution.\\
\\ \hspace*{\fill} \\
In conclusion, we have proved that $u,v \in [0]_3$ is the only solution.\\
\\ \hspace*{\fill} \\
\end{document}