\documentclass{article}
\usepackage{graphicx}
\usepackage{float}
\usepackage{subfigure} 
\usepackage{amsmath}
\usepackage{amssymb}

\author{Wenye Xiong 2023533141}
\title{Homework 1}
\begin{document}
\maketitle

\section{Problem 1}
\\ \hspace*{\fill} \\
\raggedright
As 11 is a prime number, $Z^*_{11} = {1,2,3,4,5,6,7,8,9,10}$
\\ \hspace*{\fill} \\
We can find the modular inverse using the Extended Euclidean Algorithm.\\
\\ \hspace*{\fill} \\
For example, to find the modular inverse of 3, we have:\\
\\ \hspace*{\fill} \\
\centering
\begin{pmatrix}
11 & 3 \\
1 & 0\\
0 & 1\\
\end{pmatrix}
\rightarrow
\begin{pmatrix}
2 & 3 \\
1 & 0\\
-3 & 1\\
\end{pmatrix}
\rightarrow
\begin{pmatrix}
2 & 1 \\
1 & -1\\
-3 & 4\\
\end{pmatrix}
\\ \hspace*{\fill} \\

And we get -1*11+4*3=1.\\
\\ \hspace*{\fill} \\
\raggedright
So we can find the modular inverse of 3 is 4.\\
\\ \hspace*{\fill} \\
We can repeat this process for other elements in $Z^*_{11}$ to find their modular inverses.\\
\\ \hspace*{\fill} \\
After all, we have the following table:\\
\begin{table}[H]
\centering
\begin{tabular}{|c|c|}
\hline
a & $a^{-1}$\\
\hline
1 & 1\\
2 & 6\\
3 & 4\\
4 & 3\\
5 & 9\\
6 & 2\\
7 & 8\\
8 & 7\\
9 & 5\\
10 & 10\\
\hline
\end{tabular}
\end{table}

\\ \hspace*{\fill} \\

\end{document}